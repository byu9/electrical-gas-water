\documentclass{article}
\title{Implementation Report}
\usepackage{listings}
\lstset{basicstyle=\ttfamily\small,breaklines=true}

\begin{document}
\maketitle




\section{Overview}
The convexified optimization problem was instantiated for a 3-part integrated
system consisting of separate electrical, gas, and water components.

The IEEE 13-feeder test system with simulated time series for load and renewable
energy resources was selected as the electrical component. The GasLib dataset
with simulated demand data was selected as the gas component. A modified system
adapted from an EPANET2 sample network with simulated water demand was selected
for the water component.

The coupling elements included 1 gas generator, 1 power-to-gas facility, and 1
water pump. Wholesale electricity provided by the power grid through a
substation node was implemented as a conventional generator with a linear cost
function.

The scheduling horizon contained 2 hours.




\section{Platform}
The results were obtained on a platform running a Linux distribution with kernel
version 5.10.92-1 on an Intel Skylake CPU with 4 hardware threads, a maximum
clock frequency of 2.30 GHz, and 3072 KB of level-1 cache per thread. The
platform has around 15 GB of available DDR4 physical memory. The solver package
was Gurobipy 9.5.0 build v9.5.0rc5 with an academic lincese. The program was
executed by Python interpreter version 3.9.2.

An academic license was used since the number of decision variables exceeded the
limitation of the free license.




\section{Statistics}
The instantiation contained 2 quadratic objective terms, in total of 250
decision variables, of which 248 were continuous and 2 were binary, 260 linear
constraints, and 68 quadratic constraints. The solution was obtained after 23
iterations in a total of 0.03 seconds.




\section{Issues and Workarounds}

\subsection{Units of Measure}
The model involved 3 different systems each having their individual convention
for units of measure. Converting quantities into SI units in favor of uniformity
were initially considered but dismissed during a discussion due to the following

\begin{enumerate}
\item The datasets were primarily imperial.
\item SI units could cause excessively small or large constants, resulting in an
  ill-conditioned matrix in the solver that may exacerbate the loss of precision
  in IEEE-754 based floating point arithmetics.
\end{enumerate}

\paragraph{Workarounds}
The chosen workaround was to select a list of units that creates constants with
reasonable magnitudes and employ unit conversions extensively to ensure the the
meaningfulness of the results.

An alternative workaround to address the unit-related numeric issue is to use an
SI-based per-unit system for the electrical, gas, and water components, for
which the suitable bases are selected such that the magnitude of the resulting
quantities are around 1. This approach was not adopted since the details of the
components were not initially clear during the early stages of implementation,
and out of the fear that the per-unit system would further complicate the
problem and obstruct the diagnosis of potential defects.


\subsection{Big-M Method}
The constant $M$ used in constraints formulated using the big-M method resulted
in an ill-conditioned matrix in the solver that may exacerbate the loss of
precision in IEEE-754 floating point arithmetics.

\paragraph{Workaround}
The big-M method were only used for constraining the water flow to the on-off
status of the pump. The constant $M$ was selected such that it was larger than
the all higher bounds of the water flow in pump-enabled pipelines, so that the
flows are not affected by the arbitrary selection of $M$, but not large enough
to excessively increase the condition number of the matrix in the solver.


\subsection{Sampling Frequency}
The original mathematical model was based on the assumption that the scheduling
horizon and all time series data had a uniform sampling rate of 1 (unit of
time), since a number of equations implicitly multiplied by 1 (unit of time).

\paragraph{Workaround}
A clause was added in the Contract of Implementation that designates the
scheduling inteval to be 1 hour and units of all quantities involving
multiplication or division by time to be hour-based.


\subsection{Fields of Definition of Symbol Sets}
The fields of definition of some symbol sets were unclear or unspecified.

\paragraph{Workaround}
A Contract of Implementation was created to clarify the unspecified details in
the model. The contract was periodically reviewed to ensure state of coherency.



\subsection{Feasibility of the Water Component}
The water component, when isolated from the rest of the system, was
infeasible. It was found that after the removal of a water pump on line 7-8, the
water model contained insufficient degrees of freedom. The constraint set
resulted in an empty feasibility space.

\paragraph{Workaround}
The dataset for the water component was modified so that the topology was
operative under a single pump.


\section{Results}
\lstinputlisting{results.txt}


\end{document}
